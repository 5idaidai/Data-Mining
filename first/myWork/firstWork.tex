%!Tex Program = xelatex
%%%%%%%%%%%%%%%%%%%%%%%%%%%%%%%%%%%%%%%%%
% Thin Sectioned Essay
% LaTeX Template
% Version 1.0 (5/2/14)
%
% This template has been downloaded from:
% http://www.LaTeXTemplates.com
%
% Original Author:
% Nicolas Diaz (nsdiaz@uc.cl) with extensive modifications by:
% Vel (vel@latextemplates.com)
%
% Modifications by:
% Utopiar (Utopiar@gs.zzu.edu.cn) In ZZUNLP
%
% License:
% CC BY-NC-SA 3.0 (http://creativecommons.org/licenses/by-nc-sa/3.0/)
%
%%%%%%%%%%%%%%%%%%%%%%%%%%%%%%%%%%%%%%%%%

%----------------------------------------------------------------------------------------
%	引入支持包和文档设置
%----------------------------------------------------------------------------------------

\documentclass[a4paper, 11pt, hyperref, titlepage]{article}
\usepackage{xltxtra}
\usepackage{fontspec}
\setmainfont[Mapping=tex-text]{宋体}
\setsansfont{黑体}

\newcommand{\tabincell}[2]{\begin{tabular}{@{}#1@{}}#2\end{tabular}}

\usepackage[protrusion=true,expansion=true]{microtype} % Better typography
\usepackage{graphicx} % 图片包
\usepackage{wrapfig} % 允许图片或绕

\usepackage{float} %图片位置
\usepackage{amsmath} %公式
\usepackage{bm}  %加粗
\usepackage{CJK} %字体包
\usepackage{multirow}

%中文换行
\XeTeXlinebreaklocale "zh"
\XeTeXlinebreakskip = 0pt plus 1pt

\usepackage{indentfirst}%段落首行缩进
\setlength{\parindent}{2em}%缩进两个字符

\usepackage[colorlinks,linkcolor=black,anchorcolor=black,citecolor=black]{hyperref}  %超链接包

\makeatletter
%\renewcommand\@biblabel[1]{\textbf{#1.}} % 更换引用序号'[1]' 为'1.'
\renewcommand{\@listI}{\itemsep=0pt} % 减少逐项列举项目之间的空间,并列举环境和参考书目
\renewcommand\refname{参考文献} %定义为中文参考文献

%黑体
\newcommand\fontnamehei{黑体}
\newfontinstance\HEI{\fontnamehei}
\newcommand{\hei}[1]{{\HEI #1}}

%更改行间距
\linespread{1.5}

\renewcommand{\maketitle}{ % 自定义标题 - 在这里并不修改标题和作者的名字

\begin{flushleft} % 右对齐
{\LARGE\@title} % 加大标题字体

\vspace{50pt} % 标题和作者间的间距

{\large\@author} % 作者名格式
\\\@date % Date

\vspace{40pt} % 作者和摘要的距离
\end{flushleft}
}

%----------------------------------------------------------------------------------------
%	TITLE
%----------------------------------------------------------------------------------------

\title{\textbf{数据挖掘导论}\\ % Title
第一次作业} % Subtitle

\author{\textsc{梁军} % Author
\\{\textit{郑州大学}}} % Institution

\date{\today} % Date

%----------------------------------------------------------------------------------------

\begin{document}\pagestyle{empty}
\vspace*{3cm}

\pagestyle{headings}

\maketitle % Print the title section
\newpage



%----------------------------------------------------------------------------------------
%	ESSAY BODY
%----------------------------------------------------------------------------------------

\section{\hei{Wine Quality数据集}}

酒质量(Wine Quality)数据集是由葡萄牙米尼奥大学(Univ. Minho)P. Cortez,和CVRVV(一直致力于提高葡萄牙青酒品质的组织)的A. Cerdeira, F. Almeida, T. Matos and J. Reis 于2009年创建的\cite{cortez2009modeling}。该数据集中所涉及的是葡萄牙青酒——一种产自葡萄牙西北地区米尼奥的独特品种。这种酒占葡萄牙酒业产量的15\%,其中约10\%用于出口。该数据集采集的数据是葡萄牙青酒中两种常见的品种:白葡萄酒和红葡萄酒(按照区域划分)。数据是由CCRVV 在2004年5月到2007年2月从受保护的原产地采集测试样品并测量获取的。这些数据是通过一个叫做iLab的自动化系统记录的,它会自动管理的葡萄酒样品的测试生产要求和实验室测量、葡萄酒口感分析的过程。每个葡萄酒的口感是由三个品酒师一起评价,然后给出一个0-10的评分(0表示最坏,10表示最好),然后从三个打分中选取一个中间值作为该葡萄酒的质量得分。
%--------------------------------------------------------------------------------------------
\subsection{\hei{数据描述}}

该数据集包含1599个红葡萄酒的信息,4898个白葡萄酒的信息。每种酒的特征用下面11中属性描述:

\begin{enumerate}
\setlength{\baselineskip}{0.5\baselineskip}{\item fixed acidity(非挥发性酸含量),单位:$(g(tartaric \quad acid)/dm^3)$
    \item volatile acidity(挥发性酸含量),单位:$(g(tartaric \quad acid)/dm^3)$
    \item citric acid(柠檬酸含量),单位:$(g/dm^3)$
    \item residual sugar(香槟酒甜度),单位:$(g/dm^3)$
    \item chlorides(氯化物含量),单位:$(g(tartaric \quad acid)/dm^3)$
    \item free sulfur dioxide(游离二氧化硫含量),单位:$(mg/dm^3)$
    \item total sulfur dioxide(二氧化硫总含量),单位:$(mg/dm^3)$
    \item density(密度),单位:$(g/cm^3)$
    \item pH(pH值)
    \item sulphates(硫酸盐含量),单位:$(g(potassium \quad sulphate)/dm^3)$
    \item alcohol(酒精度),单位:$(\%vol.)$
}
\end{enumerate}


\begin{center}
% Table generated by Excel2LaTeX from sheet 'Sheet1'
\begin{tabular}{lrrrrrr}
\hline
\multicolumn{ 1}{c}{{\bf 属性}} &     \multicolumn{ 3}{c}{{\bf 红葡萄酒}} &     \multicolumn{ 3}{c}{{\bf 白葡萄酒}} \\

\multicolumn{ 1}{c}{} &  {\bf 最小值} &  {\bf 最大值} &  {\bf 平均值} &  {\bf 最小值} &  {\bf 最大值} &  {\bf 平均值} \\
\hline
fixed acidity &        4.6 &       15.9 &        8.3 &        3.8 &       14.2 &        6.9 \\

volatile acidity &        0.1 &        1.6 &        0.5 &        0.1 &        1.1 &        0.3 \\

citric acid &          0 &          1 &        0.3 &          0 &        1.7 &        0.3 \\

residual sugar &        0.9 &       15.5 &        2.5 &        0.6 &       65.8 &        6.4 \\

 chlorides &       0.01 &       0.61 &       0.08 &       0.01 &       0.35 &       0.05 \\

free sulfur dioxide &          1 &         72 &         14 &          2 &        289 &         35 \\

total sulfur dioxide &          6 &        289 &         46 &          9 &        440 &        138 \\

   density &       0.99 &      1.004 &      0.996 &      0.987 &      1.039 &      0.994 \\

        pH &        2.7 &          4 &        3.3 &        2.7 &        3.8 &        3.1 \\

  sulphate &        0.3 &          2 &        0.7 &        0.2 &        1.1 &        0.5 \\

   alcohol &        8.4 &       14.9 &       10.4 &          8 &       14.2 &       10.4 \\
\hline
\end{tabular}

\end{center}

\subsection{\hei{属性类型}}

属性是对象的性质或特性,它因对象而已,或随时间而变化。属性的类型有以下四种\cite{FanMing2006}:

\begin{enumerate}
\setlength{\baselineskip}{0.5\baselineskip}{
    \item 标称:仅仅表示不同的名字;
    \item 序数:提供足够的信息确定队形的序;
    \item 区间:对于区间属性,值之间的差是有意义的,即存在测量单位;
    \item 比率:对于比率变量,差和比率都是有意义的。
}
\end{enumerate}
根据以上定义,我们可以将用于鉴定酒质量的11个属性分别划到对应的属性类型:
\begin{enumerate}

    \item 比率型数据:大多数属性都属于此类型数据,如:volatile acidity(挥发性酸含量),citric acid(柠檬酸含量),residual sugar(香槟酒甜度),chlorides(氯化物含量),free sulfur dioxide(游离二氧化硫含量),total sulfur dioxide(二氧化硫总含量),density(密度),sulphates(硫酸盐含量)和alcohol(酒精度)。因为对这些属性进行比率运算是有意义的,所以它们都属于比率型数据。

    \item 序数型数据:酒的品质应该属于序数型数据,虽然用1-10来表示酒的品质好坏,单这仅仅是定义了一个等级,并没有测量单位,进行加法和减法运算是没有意义的,因此属于序数型数据。

    \item 区间型数据?:最初感觉pH值应该属于区间型数据,和摄氏温度类似,都是定义了一个基准:摄氏温度是将冰点定为0°C,而pH值是将中性溶液指通常情况下(25°C、298K左右),pH值为7.0的溶液,常见的有氯化钠溶液、纯水定为标准溶液。但两者也有不同,比如摄氏温度其实是定义了0°C和100°C,然后将这个区间的值等分100份,这跟区间型的定义不谋而合,而pH值却不是这样,仅仅是定义了一个基准而已,因此,最终将pH 值也划归为比率型数据。

\end{enumerate}

\subsection{\hei{数据集特性}}
根据\cite{FanMing2006}中描述,我们对数据集的三个特性:维度、稀疏性和分辨率进行讨论。首先,我们来看数据集的维度,数据集的维度是数据集中的对象具有的属性数目。而对于本数据集来说,每个数据对象具有的属性数目是11,也就是该数据集的维度是11,属于低维数据,因而不存在维灾难的问题,不需要进行维规约,但是由于仅仅只有11 维可能会造成数据间的区别度不够;其次,我们来看数据的稀疏性问题,从数据集上我们可以清楚的看到,该数据集中几乎没有属性值为0的情况,也就是说该数据集不是稀疏的,对于某些仅适合处理稀疏数据的数据挖掘算法可能不适用;最后,我们看看数据的分辨率,由于数据的详细说明中并没有给出数据收集的间隔,仅告诉我们该数据集是从04年5月份到07年2月份采集的,由此可以看到该数据集的跨度是非常大的,而对象仅仅有几千个,这可能会造成数据的分辨率太低,不能有效识别出数据中的模式。

\subsection{\hei{数据集类型}}
根据\cite{FanMing2006}中的介绍,我们知道数据集的类型可以分为以下几种:

\begin{enumerate}
\setlength{\baselineskip}{0.5\baselineskip}{
    \item 记录数据
        \begin{enumerate}
            \item 事务数据和购物篮数据
            \item 数据矩阵
            \item 稀疏数据矩阵
        \end{enumerate}
    \item 基于图形的数据
        \begin{enumerate}
            \item 带有对象之间联系的数据
            \item 带有图形对象的数据
        \end{enumerate}
    \item 有序数据
        \begin{enumerate}
            \item 时序数据
            \item 序数数据
            \item 时间序列数据
            \item 空间数据
        \end{enumerate}
}
\end{enumerate}
根据数据集的特点,我们可以很容易判断出Wine Quality数据集是属于记录数据中的数据矩阵,它是一种标准的数据格式,可以使用标准的矩阵操作对数据进行变换和处理。


\section{\hei{University数据集}}

\subsection{\hei{数据描述}}
University数据集是Lebowitz M. 的一篇发表于机器学习上的论文中所使用的数据集\cite{lebowitz1984concept}。是关于大学的数据集,共有285个数据对象,部分数据对象的某些属性值有缺失,对象的属性描述如下:

\begin{enumerate}
\setlength{\baselineskip}{0.5\baselineskip}{
    \item University-name(大学名称)
    \item State(学校所在地)
    \item location(城市规模)
    \item Control(学校性质,如:私立)
    \item number-of-students(学生数量)
    \item male:female (ratio)(男女比例)
    \item student:faculty (ratio)(学生与教职工比例)
    \item sat-verbal(sat英语成绩)
    \item sat-math(sat数学成绩)
    \item expenses(费用)
    \item percent-financial-aid(助学金比例)  
    \item number-of-applicants(申请人数)
    \item percent-admittance(通过率)
    \item percent-enrolled(入学率)
    \item academics(学术规模)
    \item social(社会规模)
    \item quality-of-life(生活质量)
    \item academic-emphasis(重点学科)
}
\end{enumerate}

% Table generated by Excel2LaTeX from sheet 'Sheet2'
\begin{tabular}{lcc}
\hline

           &    Harvard &        MIT \\
\hline

     state & massachusetts & massachusetts \\

  location &      urban &      urban \\

   control &    private &    private \\

no-of-students thous &       5-10 &         5- \\

male:female ratio &      65:35 &      75:25 \\

student:faculty ratio &       10:1 &        5:1 \\

sat verbal &        700 &        650 \\

  sat math &        675 &        750 \\

expenses thous\$ &        10+ &        10+ \\

percent-financial-aid &         60 &         50 \\

no-applicants thous &      13-17 &        4-7 \\

percent-admittance &         20 &         30 \\

percent-enrolled &         80 &         60 \\

academics scale:1-5 &          5 &          5 \\

social scale:1-5 &          3 &          3 \\

quality-of-life scale:1-5 &          4 &          3 \\

academic-emphasis &    history &   sciences \\

academic-emphasis &    biology & electrical-engineering \\

academic-emphasis & liberal-arts & mechanical-engineering \\

academic-emphasis &            & engineering \\

\hline

\end{tabular}  




\subsection{\hei{属性类型}}
参照1.2对属性类型的定义,将该数据集的属性划分到对应的属性类型:

\begin{enumerate}

    \item 标称型数据 此数据集中的标称型数据有:University-name、State、Control和academic-emphasis,这些属性仅仅是标识不同的名字,没有其他的信息来确定对象的序,因此属于标称型数据。
    \item 序数型数据 此数据集中的序数型数据有:location、academics、social和quality-of-life,这些属性的值可以供我们确定对象的序,但由于这些属性都没有测量单位,不能进行加法和减法运算,因此数据序数型数据。
    \item 比率型数据 数据集中的number-of-students、sat-verbal、sat-math、expenses和number-of-applicants都可以进行差和比率预算,因此属于比率型数据;对于数据集中的male:female、student:faculty、percent-financial-aid、percent-admittance和percent-enrolled也可以进行比率运算,但由于只是百分比,自身没有测量单位,对是否属于比率型数据有点疑问?

\end{enumerate}

\subsection{\hei{数据集特性}}
根据\cite{FanMing2006}中描述,我们对数据集的三个特性:维度、稀疏性和分辨率进行讨论。首先,我们来看数据集的维度,数据集的维度是数据集中的对象具有的属性数目。而对于本数据集来说,每个数据对象具有的属性数目是18,也就是该数据集的维度是18,属于低维数据,因而不存在维灾难的问题,不需要进行维规约,但是由于仅仅只有18 维可能会造成数据间的区别度不够;其次,我们来看数据的稀疏性问题,从数据集上我们可以看到该数据集中也几乎没有属性值为0的情况(极少对象的属性值缺失),也就是说该数据集不是稀疏的,对于某些仅适合处理稀疏数据的数据挖掘算法可能不适用;最后,我们来看数据集的分辨率问题,美国大学数量大概有3000多所,本数据集中收集了近300所大学的信息,可以说已经收集了足够多的数据信息。

\subsection{\hei{数据集类型}}
参照1.4对数据集类型的定义,该数据集属于记录型数据,每个大学对象相当于一条记录,每个记录包含固定的数据字段(属性)及。记录之间或数据字段之间没有明显的联系,并且每个记录(对象)具有相同的属性集。
%----------------------------------------------------------------------------------------
%	BIBLIOGRAPHY
%----------------------------------------------------------------------------------------

%-------------------------------------------------------------------------------------------------------
%使用bibtex文件
%-------------------------------------------------------------------------------------------------------

\bibliographystyle{unsrt}

\bibliography{cite}
%-------------------------------rogers2011first---------------------------------------------------------
\end{document}
